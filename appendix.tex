\appendix
\section{Some examples}

The following illustrate some cases of stable patches that are recognized
only by PatchNet.

\begin{lstlisting}[language=diff]
commit 5567e989198b5a8d78f9b5868e48fc9f4726bdd5
Author: Madalin Bucur <...>
Date:   Mon Jun 19 18:04:16 2017 +0300

    fsl/fman: propagate dma_ops
    
    Make sure dma_ops are set, to be later used by the Ethernet driver.
    
    Signed-off-by: Madalin Bucur <...>
    Signed-off-by: David S. Miller <...>

diff --git a/drivers/net/ethernet/freescale/fman/mac.c b/...
index 0b31f8502ada..6e67d22fd0d5 100644
--- a/drivers/net/ethernet/freescale/fman/mac.c
+++ b/drivers/net/ethernet/freescale/fman/mac.c
@@ -625,0 +626,2 @@
+       set_dma_ops(&pdev->dev, get_dma_ops(priv->dev));
+

commit 2e31b4cb895ae78db31dffb860cd255d86c6561c
Author: Trond Myklebust <...>
Date:   Tue Jun 27 17:40:50 2017 -0400

    NFSv4.1: nfs4_callback_free_slot() cannot call nfs4_slot_tbl_drain_complete()
    
    The current code works only for the case where we have exactly one slot,
    which is no longer true.
    nfs4_free_slot() will automatically declare the callback channel to be
    drained when all slots have been returned.
    
    Signed-off-by: Trond Myklebust <...>

diff --git a/fs/nfs/callback_xdr.c b/fs/nfs/callback_xdr.c
index c14758e08d73..390ac9c39c59 100644
--- a/fs/nfs/callback_xdr.c
+++ b/fs/nfs/callback_xdr.c
@@ -756 +755,0 @@
-       nfs4_slot_tbl_drain_complete(tbl);
\end{lstlisting}

The following illustrate some cases of stable patches that are not recognized
by PatchNet, but are recognized by at least one of the baselines.

\begin{lstlisting}[language=diff]
commit 03f219041fdbeb31cecff41bb1cb4e1018f9cf75
Author: Luis Henriques <lhenriques@suse.com>
Date:   Wed May 17 12:21:07 2017 +0100

    ceph: check i_nlink while converting a file handle to dentry
    
    Converting a file handle to a dentry can be done call after the inode
    unlink.  This means that __fh_to_dentry() requires an extra check to
    verify the number of links is not 0.
    
    The issue can be easily reproduced using xfstest generic/426, which does
    something like:
    
        name_to_handle_at(&fh)
        echo 3 > /proc/sys/vm/drop_caches
        unlink()
        open_by_handle_at(&fh)
    
    The call to open_by_handle_at() should fail, as the file doesn't exist
    anymore.
    
    Link: http://tracker.ceph.com/issues/19958
    Signed-off-by: Luis Henriques <lhenriques@suse.com>
    Reviewed-by: "Yan, Zheng" <zyan@redhat.com>
    Signed-off-by: Ilya Dryomov <idryomov@gmail.com>

diff --git a/fs/ceph/export.c b/fs/ceph/export.c
index e8f11fa565c5..7df550c13d7f 100644
--- a/fs/ceph/export.c
+++ b/fs/ceph/export.c
@@ -93,0 +94,4 @@
+               if (inode->i_nlink == 0) {
+                       iput(inode);
+                       return ERR_PTR(-ESTALE);
+               }

commit 56199016e8672feb7b903eda003a863d5bf2b8c4
Author: Yan, Zheng <zyan@redhat.com>
Date:   Thu Jun 1 16:44:53 2017 +0800

    ceph: use current_kernel_time() to get request time stamp
    
    ceph uses ktime_get_real_ts() to get request time stamp. In most
    other cases, current_kernel_time() is used to get time stamp for
    filesystem operations (called by current_time()).
    
    There is granularity difference between ktime_get_real_ts() and
    current_kernel_time(). The later one can be up to one jiffy behind
    the former one. This can causes inode's ctime to go back.
    
    Signed-off-by: "Yan, Zheng" <zyan@redhat.com>
    Signed-off-by: Ilya Dryomov <idryomov@gmail.com>

diff --git a/fs/ceph/mds_client.c b/fs/ceph/mds_client.c
index f38e56fa9712..0c05df44cc6c 100644
--- a/fs/ceph/mds_client.c
+++ b/fs/ceph/mds_client.c
@@ -1690 +1689,0 @@
-       struct timespec ts;
@@ -1709,2 +1708,2 @@
-       ktime_get_real_ts(&ts);
-       req->r_stamp = timespec_trunc(ts, mdsc->fsc->sb->s_time_gran);
+       req->r_stamp = timespec_trunc(current_kernel_time(),
+                                     mdsc->fsc->sb->s_time_gran);
\end{lstlisting}
