\begin{abstract}
% This document is a model and instructions for \LaTeX.
% This and the IEEEtran.cls file define the components of your paper [title, text, heads, etc.]. *CRITICAL: Do Not Use Symbols, Special Characters, Footnotes, 
% or Math in Paper Title or Abstract.

%Linux kernel stable versions serve the needs of users who value stability
%of the kernel over new features. The quality of such stable kernel versions
%depends on the initiative of kernel maintainers to propagate bug fixing
%patches to the stable versions. Thus, it is desirable to consider to what
%extent this process can be automated. A previous approach
%% based on Learning from Positive and Unlabeled Examples (SVM) and Support Vector Machine (SVM) 
%relies on words from commit messages and a small set of manually constructed code
%features. This approach, however, shows only moderate accuracy.
%In this paper, we investigate whether deep learning can provide a more
%accurate solution. We propose PatchNet, a hierarchical deep learning-based approach that is capable of automatically extracting features from commit messages and commit code and using them to identify stable patches. PatchNet contains a deep hierarchical structure that mirrors the hierarchical and sequential structure of commit code, making it distinctive from the existing deep learning models on source code.
%% \ty{I add this sentence highlighting the novelty of PatchNet. James,pls verify this.} 
%Experiments on 82,403 recent Linux patches confirm the superiority of   
%PatchNet against various state-of-the-art baselines, including the one actively-used by Linux kernel maintainers. 
% The results show that PatchNet surpasses the best-performing baseline by 6.55\%, 7.8\%, and 6.3\% in terms of accuracy, F1, and AUC, respectively. 
% 6.55\%, 0.12\%, 16.13\%, 7.8\%, and 6.3\%,\ty{not consistent with the ones reported in experiment section} in terms of accuracy, precision, recall, F1, and AUC, respectively. 

% Linux kernel stable versions serve the needs of users who value stability of the kernel over new features. The quality of such stable kernel versions depends on the initiative of kernel maintainers to propagate bug fixing patches to the stable versions. Thus, it is desirable to consider to what extent this process can be automated. In this work, we propose PatchNet, an automated tool based on a hierarchical deep learning-based approach to extract features from commit messages and commit code and using them to identify stable patches. PatchNet contains a deep hierarchical structure that mirrors the hierarchical and sequential structure of commit code, making it distinctive from the existing deep learning models on the source code. PatchNet accepts as input a set of patches to train a predicted model. We then apply the predicted model on a new set of patches to collect a list of predicted scores of given patches used to estimate how likely the given patches are bug fixing patches. Noticeably, our implementation of PatchNet provides several options allowing users to select parameters for the training process. Moreover, our tool is applicable to 

Identifying bug fixing patches is an important problem as it serves the needs of users who want to take advantages of the latest features. Thus, it is desirable to consider to what extent this process can be automated. However, writing an application used to solve this problem is time-consuming and requires skills from developers. In this work, we propose PatchNet, an automated tool based on a hierarchical deep learning-based approach to extract features from commit messages and commit code and using them to identify stable patches. PatchNet contains a deep hierarchical structure that mirrors the hierarchical and sequential structure of commit code, making it distinctive from the existing deep learning models on the source code. PatchNet accepts as input a set of patches to train a predicted model. We then apply the predicted model on a new set of patches to collect a list of predicted scores of given patches used to estimate how likely the given patches are bug fixing patches. Noticeably, our implementation of PatchNet provides several options allowing users to select parameters for the training process. Moreover, the tool is also applicable to other problems in software engineering domains such as defect prediction, bug localization, etc. 
\end{abstract}
